\documentclass
[ 12pt,
  parskip=half % Absätze durch einen vergrößerten Zeilenabstand getrennt
]{scrreprt}

\usepackage{mathpazo} %glourious palatino als Schriftart
\usepackage{beramono} %tt
\usepackage[scaled=0.95]{berasans} %ss

\usepackage[a4paper,% Papierdimensionen festlegen
inner=2.0cm, outer=2.0cm, bindingoffset=0.5cm,%
top=1.5cm,bottom=1.5cm, %
footskip=1.0cm,includeheadfoot]{geometry}

\usepackage[utf8]{inputenc} % Input encoding (allow direct use of special characters like "ä")
\usepackage[ngerman]{babel} % Sprache auf Deutsch
\usepackage[T1]{fontenc} % Schöneres Inhaltsverzeichnis
%\usepackage{quotchap} % Beautiful chapter decoration
\usepackage{easyfig} % Bilder einfach einbinden
\usepackage{hyperref} % Links auf anderes Stellen im Dokument, PDF-Lesezeichen
\usepackage{tabularx} % Auto Spaltenbreite und Zeilenumbruch in Tabellen mit |X|
\usepackage{tabulary} % Auto Spaltenbreite und Zeilenumbruch in Tabellen mit |L|R|C|J|
\usepackage{xcolor} % Für Farben
\usepackage{lipsum} % Für 'Lorem ipsum...' mit \lipsum[<Anzahl Absätze>]
\usepackage[colorinlistoftodos]{todonotes} % Pretty Todo Notes & List of Todos
\usepackage{datetime} % Zur Ausgabe des aktuellen Datums
\usepackage[chapter]{minted} % Multi-Color-Listings
\usepackage{caption} % Minted: Für Bildunterschriften bei Listings
\usepackage{ulem} % Text unterstreichen (\uline) und durchstreichen (\sout)
\usepackage[binary-units=true]{siunitx} % SI-EInheiten

%%%%%%%%%% Farben
\definecolor{darkblue}{rgb}{0, 0, 0.5}
\definecolor{darkgreen}{rgb}{0, 0.5, 0}
\definecolor{darkred}{rgb}{0.7, 0, 0}
\definecolor{darkyellow}{rgb}{0.9, 0.6, 0}
\definecolor{gray}{rgb}{0.5, 0.5, 0.5}

%%%%%%%%%% Konstanten

\newcommand{\titel}{ReConf}
\newcommand{\untertitel}{}
\newcommand{\authorname}{Niklas Kühl}
\newcommand{\thesisname}{}
%\newcommand{\Datum}{20. Februar 2017}
\newcommand{\Datum}{\the\day. \monthname{} \the\year} % Aktuelles Datum

\newcommand{\pdfdelta}{\texorpdfstring{$\delta$}{Delta}}
	
%%%%%%%%%% Befehle
%\newcommand{\todo}[1]{\textcolor{red}{TODO: #1}} % Todo ohne Package todonotes
\newcommand{\todoinl}[1]{\todo[inline]{#1}}
\newcommand{\todoPrioHigh}[1]{\todo[color=red!40, inline]{#1}}
\newcommand{\todoPrioNormal}[1]{\todo[color=yellow!40, inline]{#1}}
\newcommand{\todoPrioLow}[1]{\todo[color=green!40, inline]{#1}}

\newcommand{\textGray}[1]{\textcolor{gray}{#1}}

%%%%%%%%%% Konfiguration hyperref
\hypersetup{ 
	colorlinks=true,
	breaklinks=true, % Erlaubt Links, über mehrere Zeilen zu gehen
	linkcolor=darkblue, 	% Farbe aller Links auf andere Stellen im Dokument: Inhaltsverzeichnis, Glossar, Abkürzungen
	urlcolor=blue,
	citecolor=darkgreen,
	pdftitle={\thesisname},
	pdfsubject={\titel},
	pdfauthor={\authorname},
}


%%%%%%%%%% Konfiguration minted
\usemintedstyle[rust]{colorful} % Style. Siehe https://www.sharelatex.com/learn/Code_Highlighting_with_minted#Reference_guide
\renewcommand\listoflistingscaption{Quellcodeverzeichnis}
\setminted{
	frame=single, 
	breaklines=true, 
	fontsize=\small,
	tabsize=2, 
	linenos
}


% Globales Escapeinside
%\setminted[rust]{escapeinside=\%\%}

% Listing Umgebung, um caption hinzuzufügen
\newenvironment{mylisting}[1][H]
{\captionsetup{aboveskip=-0.2\normalbaselineskip}\begin{listing}[#1]}
{\end{listing}}

%%%%%%%%%%%%%%%%%%%%%%%%%%%%%%%%%%%%%%%%%%%%%%%%%%%%%%%%%%%
\begin{document}

\pagenumbering{Roman}

%%%%%%%%%%%%%%%%%%%%%%%%%%%%%%%%%%%%%%%%%%%%%%%%%%%%%%%%%%%
\titlehead{
	\centering
	\bigskip
}

\subject{\thesisname}
\title{\titel}
\subtitle{\Large \untertitel}
\date{\vspace{-1cm}\Datum}

\publishers{
	\vfill
	\normalsize
	\raggedright
	{\small Eingereicht von:}\\
	\smallskip
	{\Large \authorname}\\
	inf102861@fh-wedel.de\\
	

}

\maketitle

%%%%%%%%%%%%%%%%%%%%%%%%%%%%%%%%%%%%%%%%%%%%%%%%%%%%%%%%%%%
\tableofcontents
\listoffigures
\listoftables
\listoflistings
\cleardoublepage
\pagenumbering{arabic}

%%%%%%%%%%%%%%%%%%%%%%%%%%%%%%%%%%%%%%%%%%%%%%%%%%%%%%%%%%%
\chapter{Sollte erklärt worden sein}

- Inhalt der Matrizen: Zahlen, Signed Float bzw. Fixed
- Inhalt von y train: Zahlen, Klassen der Daten, Unsigned Integer

\chapter{Benutzerhandbuch}

\section{Funktionsumfang}
\todoinl{kann evtl weg}

Auf dem FPGA ist ein Algorithmus für ein Neuronales Netz implementiert. Die Steuerung des Algorithmus erfolgt von einem PC über eine RS232-Schnittstelle. 

\section{Bedienungsanleitung}

\subsection{Kodierung der Matrizen}

Die Elemente der Matrizen sind 8 Bit breit. Diese 8 Bit werden als vorzeichenbehafteter Festkommawert interpretiert. Es werden 4 Bit für das Vorzeichen und den Ganzzahlanteil benutzt. Der Gebrochene Anteil benutzt weitere 4 Bit. Der Wertebereich der Elemente beträgt somit $7.9375$ bis -$8.0000$. Der kleinste von $0.0$ verschiedene Wert ist $\pm0.0625$.

\subsection{Kommunikation zwischen FPGA und PC}
\label{sec:Kommunikation-zwischen-FPGA-und-PC}

Die serielle Kommunikation zwischen FPGA und PC läuft mit einer Baudrate von 9600, 1 Stoppbit, kein Paritätsbit. Es wird folgendes Protokoll verwendet:
\begin{enumerate}
	\item Reset des FPGA per SW17
	\item FPGA führt Initalisierung durch
	\item FPGA sendet Steuerungsbyte an PC: 0xF0
	\item PC sendet Steuerungsbyte an FPGA: 0xE0
	\item PC sendet 4096 zufällige Werte zur Initialisierung der Gewichte w1
	\item PC sendet Steuerungsbyte an FPGA: 0xE1
	\item PC sendet 640 zufällige Werte zur Initialisierung der Gewichte w2
	\item \label{enum:sel-mode} PC sendet Byte an FPGA zur Auswahl des Modus: 
	\begin{itemize}
		\item Training: 0xE2 (weiter mit \ref{enum:training})
		\item Test: 0xA1 (weiter mit \ref{enum:test})
	\end{itemize}
	\item \label{enum:training} Training
	\begin{enumerate}
		\item PC sendet 4096 Werte zur Initialisierung der Trainingsdaten
		\item PC sendet Steuerungsbyte an FPGA: 0xE3
		\item PC sendet 64 Werte zur Initialisierung der Klassen y\_train
		\item FPGA führt Algorithmus zum Training des Neuronalen Netzes aus
		\item FPGA sendet Steuerungsbyte an PC: 0xF1
		\item Zurück zu \ref{enum:sel-mode}
	\end{enumerate} 
	\item \label{enum:test} Test
	\begin{enumerate}
		\item PC sendet 4096 Werte zur Initialisierung der Testdaten
		\item FPGA führt Algorithmus zum Test der Daten aus
		\item FPGA überträgt die Bewertung der Testdaten (640 Bytes)
		\item Zurück zu \ref{enum:sel-mode}
	\end{enumerate} 
\end{enumerate}

\subsection{Bedienung des Altera DE2 Boards}

Die LEDs LEDR1 und LEDR2 dienen zur Signalisierung von Fehlern:
\begin{itemize}
	\item LEDR1: Leuchtet, wenn ein Protokollfehler aufgetreten ist
	\item LEDR2: Leuchtet, wenn ein Fehler bei der seriellen Kommunikation aufgetreten ist
\end{itemize}

Der Schalter SW17 dient zum zurücksetzten des Systems.

Der Schalter SW0 dient zum aktivieren des Debug-Modus, über welchen interne Daten des Systems ausgelesen werden können. Das Verlassen des Debug-Modus ist nur mit einem Reset möglich. Eine genaue Beschreibung der Funktionsweise erfolgt in Kapitel \todo{Referenz}.
 
Über die Sieben-Segment-Anzeige der Altera DE2 Boards wird der interne Zustand des Systems angezeigt. Die Bedeutung der Anzeige ist in Kapitel \todo{Referenz} beschrieben.

%%%%%%%%%%%%%%%%%%%%%%%%%%%%%%%%%%%%%%%%%%%%%%%%%%%%%%%%%%%
\chapter{Planung}

\section{Problemanalyse}

\subsection{Analyse des Algorithmus}

Für die Implementierung des Algorithmus auf dem FPGA wurden die Möglichkeiten zur Parallelisierung untersucht. Der Algorithmus besteht im Wesentlichen aus zwei Teilen:
\begin{enumerate}
	\item \label{enum:calc-scores} Berechnung der scores
	\item \label{enum:calc-mat} Berechnung der übrigen Matrizen
\end{enumerate} 

Die Berechnung der scores (Schritt \ref{enum:calc-scores}) läuft größtenteils sequentiell ab, weswegen eine parallele Ausführung mehrerer Operationen hier kaum Gewinn bringt.

In Schritt \ref{enum:calc-mat} werden die Matrizen w1, w2, b1 und b2 angepasst. Da diese Berechnungen größtenteils unabhängig voneinander ablaufen, ist hier eine Parallelisierung sinnvoll. Die gleichzeitige Ausführung von 2 Befehlen ist hier gut möglich. Die parallele Ausführung von 3 oder mehr Befehlen nur ist aufgrund von Datenabhängigkeiten hingegen nur sinnvoll, wenn die gleiche Operation mehrmals parallel ausgeführt werden kann.

In Tabelle \ref{table:parallel-algo} sind die Schritte des Algorithmus zur Parallelen Ausführung von 2 Operationen dargestellt.

\begin{center}
	\begin{tabulary}{\columnwidth}{|C|C|C|}
		\hline
		Schritt & Operation 0 & Operation 1\\
		\hline
		0 & d = mul(x\_train, w1) & -\\
		1 & hl = d + b1 & -\\
		2 & x\_train\_trans = x\_train\textsuperscript{T} & hl\_ReLu = max(0, hl)\\
		3 & d2 = mul(hl\_ReLu, w2) & -\\
		4 & scores = d2 + b2 & -\\
		5 & w2\_trans = w2\textsuperscript{T} & scores = max(scores, 0) \\
		6 & - & scores: richtigen Wert um 1 verringern\\
		7 & - & scores: Division durch 64\\
		8 & dhidden = mul(scores, w2-trans) & -\\
		9 & scores2 = Orientierung von scores ändern & dhidden = max(dhidden, 0)\\
		10 & dw1 = mul(x-train-trans, dhidden) & w1\_reg = w1 * reg\\
		11 & dw1 = dw1 + w1\_reg & db1 = Spaltenweise Summe von dhidden\\
		12 & dw2 = mul(hl-relu, scores2) & dw1 = -step-size * dw1\\
		13 & w1 = w1 + dw1 & w2\_reg = w2 * reg\\
		14 & dw2 = dw2 + w2\_reg & db = spaltenweise summe scores2\\
		15 & - & dw2 = -stepsize * dw2\\
		16 & w2 = w2 + dw2 & db2 = -stepsize * db2\\
		17 & b2 = b2 + db2 & db1 = -stepsize * db1\\
		18 & b1 = b1 + db1 & -\\
		\hline
	\end{tabulary}
	\label{table:parallel-algo}
	\captionof{table}{Parallele Anordnung der Algorithmus-Schritte}
\end{center}


Der Algorithmus operiert auf Matrizen, welche gespeichert werden müssen. Es müssen in jedem Fall 4 Speicherplätze (Register) für die statischen Matrizen w1, w2, b1 und b2 vorhanden sein. Zusätzliche werden allerdings Speicherplätze für temporäre Matrizen benötigt.

Durch eine Analyse wurde festgestellt, dass der Algorithmus bei der in Tabelle \ref{table:parallel-algo} dargestellten Anordnung der Schritte maximal 10 Matrix-Speicherplätze benötigt. Die Verteilung der 4 statischen Matrizen auf die Speicherplätze ist in Tabelle \ref{table:algo-regs-static} dargestellt. Die Verteilung der temporären Matrizen auf die 6 übrigen Matrix-Register ist in Tabelle \ref{table:algo-regs} dargestellt. In einer Zeile der Tabelle ist dabei der Zustand vor dem jeweiligen Algorithmus-Schritt dargestellt.

\begin{center}
	\begin{tabulary}{\columnwidth}{|C|C|}
		\hline
		Matrix & Register\\
		\hline
		w1 & Reg 0\\
		b1 & Reg 1\\
		w2 & Reg 2\\
		b2 & Reg 3\\
		\hline
	\end{tabulary}
	\label{table:algo-regs-static}
	\captionof{table}{Zuordnung der statischen Matrizen zu den Matrixregistern}
\end{center}

\begin{center}
	\begin{tabulary}{\columnwidth}{|C|C|C|C|C|C|C|C|C|C|C|}
		\hline
		Schritt & Reg 4 & Reg 5 & Reg 6 & Reg 7 & Reg 8 & Reg 9\\
		\hline
		0 & x\_train & - & - & - & - & -\\	
		1 & x\_train & - & - & d & - & -\\
		2 & x\_train & - & - & hl & - & -\\
		3 & - & x\_train\textsuperscript{T} & - & hl\_ReLu & - & -\\
		4 & - & x\_train\textsuperscript{T} & - & hl\_ReLu & d2 & -\\
		5 & - & x\_train\textsuperscript{T} & - & hl\_ReLu & scores & -\\	
		6 & w2\textsuperscript{T} & x\_train\textsuperscript{T} & - & hl\_ReLu & scores & -\\
		7 & w2\textsuperscript{T} & x\_train\textsuperscript{T} & - & hl\_ReLu & scores & -\\
		8 & w2\textsuperscript{T} & x\_train\textsuperscript{T} & - & hl\_ReLu & scores & -\\
		9 & - & x\_train\textsuperscript{T} & dhidden & hl\_ReLu & scores & -\\
		10 & - & x\_train\textsuperscript{T} & dhidden & hl\_ReLu & - & scores2\\
		11 & dw1 & - & dhidden & hl\_ReLu & w1\_reg & scores2\\
		12 & dw1 & db1 & - & hl\_ReLu & - & scores2\\
		13 & dw1 & db1 & dw2 & - & w2\_reg & scores2\\
		14 & dw1 & db1 & dw2 & - & - & scores2\\
		15 & dw1 & db1 & dw2 & db2 & - & -\\
		16 & dw1 & db1 & dw2 & db2 & - & -\\
		17 & dw1 & db1 & dw2 & db2 & - & -\\
		18 & dw1 & db1 & dw2 & db2 & - & -\\
		\hline
	\end{tabulary}
	\label{table:algo-regs}
	\captionof{table}{Zuordnung der temporären Matrizen zu den Matrixregistern}
\end{center}

\subsection{Aufteilung der RAM-Blöcke}

Da für die Matrizen sehr viel Speicher benötigt wird, müssen die Daten im RAM abgelegt werden. Für die Verwaltung des RAM gibt es zwei verschiedene Ansätze:
\begin{enumerate}
	\item Benutzung eines großen RAM-Blocks
	\item Für jede Matrix wird ein eigener, kleiner RAM-Block benutzt
\end{enumerate}

Die Nutzung eines großen RAM-Blocks besitzt den Vorteil, dass ein größerer Datenbus genutzt werden kann, wodurch eine größere Datenmenge verarbeitet werden kann. Allerdings wird die Menge der Daten, welche parallel verarbeitet werden kann, auch durch andere Faktoren begrenzt (z.B. Anzahl der HW-Multiplizierer). Dadurch ist nur eine geringe Steigerung der verarbeiteten Datenmenge möglich.

Bei Nutzung eines großen RAM-Blocks besteht auch die  Möglichkeit, flexibel Speicher zuzuweisen. Dadurch kann der Speicherverschnitt minimiert werden, welcher bei Nutzung von separaten RAM-Blöcken entsteht. Allerdings erfordert dies ein komplexes Speichermanagement, da der Zugriff auf jede Matrix unter Berücksichtigung der Start-Speicheradresse erfolgen muss. 

Die Nutzung seperater RAM-Blöcke bietet den Vorteil, dass auf jeden Block parallel zugegriffen werden kann. Dadurch ist eine wesentlich höhere Parallelisierung möglich als bei Verwendung eines großen RAM-Blocks. Bei einem großen RAM-Block ist nur der gleichzeitige Zugriff auf 2 verschiedene Speicheradressen möglich, bei Verwendung von 10 seperaten RAM-Blöcken kann hingegen auf 20 Adressen gleichzeitig zugegriffen werden. 
 
Aufgrund der einfacheren Verwendung und den wesentlich besseren Parallelisierungmöglichkeiten wird daher Variante 2 gewählt.

\subsection{Struktur der Matrizen}

Die Matrizen sollen im RAM abgelegt werden. Der verwendete FPGA besitzt 105 M4K Blöcke (also \SI{420}{\kilo\byte} Speicher). Bei 10 zu speicherndes Matrizen bleiben abgerundet auf eine 2er Potenz \SI{32}{\kilo\byte} Speicher pro Matrix. Dies erlaubt die Speicherung von 64x64 Matrizen, bei 8 Bit pro Matrixelement. 

Bei der Verwendung von 8 M4K-Blöcken (\SI{32}{\kilo\byte}) ist der Datenbus \SI{256}{\bit} breit. Dies erlaubt die gleichzeitige Verarbeitung von 32 Matrix-Elementen. Damit die Matrixoperationen möglichst schnell ausgeführt werden können, werden die Matrizen daher in Wörtern mit je 32 Elementen gespeichert. Diese Wörter können zeilen- oder spaltenweise gespeichert werden (Orientierung der Matrix). 

\subsection{Matrixoperationen}

Für den Algorithmus müssen folgende Matrixoperationen implementiert werden:
\begin{itemize}
	\item Matrixmultiplikation
	\item Matrixaddition
	\item Zeilenweise Addition einer Matrix mit einem Vektor 
	\item Matrix transponieren
	\item Elementweise Multiplikation mit Skalaren Wert
	\item Elementweise Division durch Konstante
	\item Elementweises Maximum mit Konstante 
	\item Berechnung der spaltenweisen Summe einer Matrix
	\item In jeder Zeile einer Matrix von einem bestimmten Element einen konstanten Wert subtrahieren
\end{itemize}

Die Matrixmultiplikation und die skalare Multiplikation benutzten die HW-Multiplizierer des FPGA. Es sind jeweils 32 8-Bit-Multiplizierer notwendig, da in einem Takt 32 Elemente verarbeitet werden. Da der verwendete FPGA 70 9-Bit-Multiplizierer besitzt, sind genug Ressourcen vorhanden.

\section{Systemdesign}

\subsection{Top-Level-Entity}

Die grundlegende Struktur des Gesamtsytems ist in Abbildung \ref{fig:tle} dargestellt. Die Kommunikation zwischen PC und FPGA erfolgt über eine serielle Schnittstelle. Sowohl der Zustandsautomat als auch der Matrixzugriff benutzen diese Schnittstelle. Der Zustandsautomat legt dabei fest, wann der Matrixzugriff auf die serielle Schnittstelle zugreifen darf.

Der Matrixzugriff liest bzw. schreibt einzelne Bytes über die serielle Schnittstelle und liest bzw. schreibt Matrix-Wörter in die Matrixregister des NN Algorithmus. Außerdem schreibt er Daten in den RAM-Block y\_train. Dieser wird nicht als Matrix gespeichert, da die Elemente einen anderen Datentyp besitzen.

Im Block "`NN Algorithmus"' ist der Hauptteil des Systems realisiert. Hier werden die Matrizen gespeichert und die Operationen des Algorithmus ausgeführt.

\Figure[width=\textwidth, label={fig:tle}, caption={Struktur der Top-Level-Entity}]{Grafiken/tle}

\subsection{Algorithmus}

In Abbildung \ref{fig:nn-algo} ist die Struktur des Algorithmus dargestellt. DIe Matrix-CPU stellt die Matrixregister zur Verfügung und kann Operationen auf diesen Registern ausführen. Der Programminterpreter liest die Operationen des Algorithmus und setzt die entsprechenden Eingangsports der CPU.

\Figure[width=\textwidth, label={fig:nn-algo}, caption={Struktur der Entity für den Algorithmus}]{Grafiken/nn_algo}

\subsection{Matrix-CPU}

Die Matrix-CPU
\Figure[width=\textwidth, label={fig:mat-cpu}, caption={Struktur der Matrix-CPU}]{Grafiken/mat_cpu}

\subsection{Matrix-ALU}

\Figure[width=\textwidth, label={fig:mat-alu}, caption={Struktur der Matrix-ALU}]{Grafiken/mat_alu}

\subsection{Matrixoperationen}

- haben alle einen indexgenerator
- alle heben eine einheit zur berechnung des ergebnisses
- bei den operationen, die das ergebnis elementweise berechnen, brauchen wir zusätzlich einen Matrix-Wort Zwischenspeicher
- abhängig von der Operation benötigen wir auch noch register 
	- z.b. um das zuvor berechnete element zu speichern

\Figure[width=\textwidth, label={fig:mat-mul}, caption={Struktur der Einheit zur Matrixmultiplikation}]{Grafiken/mat_mul}

\subsection{Hilfsmodule}

\Figure[width=\textwidth, label={fig:ix-gen}, caption={Struktur des Indexgenerators}]{Grafiken/ix_gen}

\Figure[width=\textwidth, label={fig:set-word-elem}, caption={Struktur der einheilt zum Elementweisen Schreiben einer Matrix}]{Grafiken/set_word_elem}



\section{Schnittstellenbeschreibung}

\subsection{NN Algorithmus}

\subsection{Matrix-CPU}

\subsection{Matrix-ALU}


%%%%%%%%%%%%%%%%%%%%%%%%%%%%%%%%%%%%%%%%%%%%%%%%%%%%%%%%%%%
\chapter{Implementierung}

\section{Erklärung wesentlicher Design-Teile}

\begin{mylisting}
	\caption{VHDL Test Listing}
	\label{listing:vhdl-test}
	\begin{minted}{vhdl}
	proc_change_state : PROCESS(p_clk_i, p_rst_i)
	BEGIN
		IF p_rst_i = '1' THEN 
			s_cur_state <= st_init;
		ELSIF rising_edge(p_clk_i) THEN
			s_cur_state <= s_next_state;
		END IF;
	END PROCESS proc_change_state;
	\end{minted}
\end{mylisting} 

\section{Beschreibung der Zustandsautomaten}

\lipsum{1}

\Figure[width=0.82\textwidth, label={fig:tle-statemachine}, caption={Zustandsautomat der Top-Level-Entity}]{Grafiken/tle_statemachine}

\lipsum{1}

\Figure[width=\textwidth, label={fig:nn-algo-statemachine}, caption={Zustandsautomat zur Ausführung des Algorithmus}]{Grafiken/nn_algo_statemachine}

%%%%%%%%%%%%%%%%%%%%%%%%%%%%%%%%%%%%%%%%%%%%%%%%%%%%%%%%%%%
\chapter{Simulation}

\section{Beschreibung der Testfälle}

\section{Beschreibung der Testergebnisse}

%%%%%%%%%%%%%%%%%%%%%%%%%%%%%%%%%%%%%%%%%%%%%%%%%%%%%%%%%%%
\chapter{Fazit}

- Eventuell werden die temporären ergebnisse w1-reg und w2-reg nicht benötigt:

Statt:

\(
w1_{reg} = w1 * reg \\
dw1' = dw1 + w1_{reg} \\
dw1'' = dw1' * stepsize \\
w1' = w1 + dw1'' \\
\)

Das entspricht ja:
\(
w1' = w1 + (dw1 + (w1 * reg))' * stepsize \\
w1' = dw1 * stepsize + w1 * (1 *reg * stepsize)
\)


Deshalb besser:

\(
w1_{tmp} = w1 * (1+stepsize*reg) \\
dw1' = dw1 * stepsize \\
w1' = w1_{tmp} + dw1' \\
\)


das spart 1 Operation und 1 Matrix Register





\end{document}